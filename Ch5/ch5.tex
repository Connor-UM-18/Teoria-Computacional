\chapter{Anexos}
\lstset{
    language=C,
    basicstyle=\ttfamily\small\color{black},
    numbers=left,
    numberstyle=\tiny,
    stepnumber=1,
    numbersep=8pt,
    backgroundcolor=\color{white},
    showspaces=false,
    showstringspaces=false,
    showtabs=false,
    frame=single,
    rulecolor=\color{magenta},
    tabsize=2,
    captionpos=b,
    breaklines=true,
    breakatwhitespace=false,
    title=\lstname,
    escapeinside={\%*}{*)},
    keywordstyle=\color{blue},
    commentstyle=\color{red},
    stringstyle=\color{orange},
    morecomment=[l][\color{cyan}]{\#},
    otherkeywords={=,!,<,>,*,+,-,&,|,^,~},
    numbers=left,                   % Coloca los números de línea a la izquierda
    numberstyle=\tiny\color{black}, % Estilo de los números de línea
    stepnumber=1,    % Incremento en el que se muestran los números de línea
    numbersep=8pt
}

\section{Palindromo.py}
Se presenta el código implementado para la solución al problema con extensión .py.\newline
\\
\begin{lstlisting}

import random

def generar_palindromo(longitud):
    reglas = {
        0: "e",
        1: "0",
        2: "1",
        3: "0P0",
        4: "1P1"
    }
    j=0
    cantidad_cadena=0
    cadena_actual=""
    cadena_vieja=""
    cambiante=0
    bandera_primera_pasada=1
    paso=1
    while j==0:
        if cantidad_cadena == longitud:
            if longitud>0:
                #Reemplazamos P por e
                regla=reglas[0]
                cadena_actual = cadena_vieja.replace("P", regla)
                guardar_en_archivo2(cadena_actual,str(0),paso)
                cantidad_cadena+=2
                paso+=1
            else:
                #condicional para cuando cadena=0
                cadena_actual= reglas[0]
                guardar_en_archivo2(cadena_actual,str(0),paso)
                paso+=1
            j=1
        elif (cantidad_cadena+2) > longitud:
            if longitud>1:
                cambiante=random.randint(1, 2)
                regla= reglas[cambiante] 
                cadena_actual = cadena_vieja.replace("P", regla)
                guardar_en_archivo2(cadena_actual,str(cambiante),paso)
                cantidad_cadena+=2
                paso+=1
            else:
                #Condicional para cuando solo tengamos entrada de (1)
                cambiante=random.randint(1, 2)
                cadena_actual= reglas[cambiante]
                guardar_en_archivo2(cadena_actual,str(cambiante),paso)
                paso+=1
            j=1
        elif (cantidad_cadena+2) == longitud:
            if longitud==2:
                cambiante=random.randint(3, 4)
                cadena_actual=reglas[cambiante]
                guardar_en_archivo2(cadena_actual,str(cambiante),paso)
                paso+=1
            else:
                cambiante=random.randint(3, 4)
                regla= reglas[cambiante]
                cadena_actual = cadena_vieja.replace("P", regla)
                guardar_en_archivo2(cadena_actual,str(cambiante),paso)
                paso+=1
            cantidad_cadena+=2
        elif (cantidad_cadena+2) < longitud:
            if bandera_primera_pasada==1:
                cambiante=random.randint(3, 4)
                cadena_actual=reglas[cambiante]
                guardar_en_archivo2(cadena_actual,str(cambiante),paso)
                bandera_primera_pasada=0
                paso+=1
            else:
                cambiante=random.randint(3, 4)
                regla=reglas[cambiante]
                cadena_actual= cadena_vieja.replace("P",regla)
                guardar_en_archivo2(cadena_actual,str(cambiante),paso)
                paso+=1
            cantidad_cadena+=2
        cadena_vieja=cadena_actual #Guardamos cadena que llevamos en la vieja.
    cadena_actual=cadena_actual.replace("e","")
    guardar_en_archivo2(cadena_actual,"e a vacio",paso)
    palindromo = cadena_actual
    return palindromo


def guardar_en_archivo(longitud, regla, palindromo):
    ruta_archivo = "C:\\Users\\soyco\\OneDrive\\Documents\\ESCOM\\sem4\\Teoria\\P2\\Pali\\output\\palindromos.txt"
    with open(ruta_archivo , "w") as archivo:
        archivo.write(f"Regla: {regla}\n")
        archivo.write(f"Longitud: {longitud}\n")
        archivo.write(f"Palindromo: {palindromo}\n\n")
        
def guardar_en_archivo2(cadena, regla, paso):
    ruta_archivo = "C:\\Users\\soyco\\OneDrive\\Documents\\ESCOM\\sem4\\Teoria\\P2\\Pali\\output\\construccion.txt"
    with open(ruta_archivo , "a") as archivo:     
        archivo.write(f"Paso: [{paso}]")
        archivo.write(f"--> (Regla {regla}):  {cadena}\n")
def main():
    opcion = input("Seleccione una opción:\n1. Ingresar longitud del palíndromo\n2. Generar automáticamente\n")
    
    if opcion == "1":
        longitud = int(input("Ingrese la longitud del palíndromo: "))
        if longitud <= 100000:
            palindromo = generar_palindromo(longitud)
            regla = "Longitud ingresada por el usuario"
            guardar_en_archivo(longitud, regla, palindromo)
            print("Palíndromo generado y guardado en el archivo 'palindromos.txt'.")
            print("Construccion del palindromo generado y guardado en el archivo 'construccion.txt'.")
        else:
            print("La longitud ingresada excede el tamaño máximo permitido.")
    elif opcion == "2":
        longitud = random.randint(1, 100000)
        palindromo = generar_palindromo(longitud)
        regla = "Generación automática"
        guardar_en_archivo(longitud, regla, palindromo)
        print("Palíndromo generado y guardado en el archivo 'palindromos.txt'.")
        print("Construccion del palindromo generado y guardado en el archivo 'construccion.txt'.")
    else:
        print("Opción inválida")

if __name__ == "__main__":
    ruta_archivo = "C:\\Users\\soyco\\OneDrive\\Documents\\ESCOM\\sem4\\Teoria\\P2\\Pali\\output\\construccion.txt"
    with open(ruta_archivo , "w") as archivo:
        print("Limpiamos archivo de construccion.")
    main()

\end{lstlisting}
\section{GraficadorLineal.py}
Se presenta el código LaTeX de este archivo mediante el siguiente link:\newline
\\
Link overleaf: $"$\url{https://www.overleaf.com/3364573574gdbzwhrghsmq}$"$\newline
Link github: $"$\url{https://www.overleaf.com/3364573574gdbzwhrghsmq}$"$\newline
